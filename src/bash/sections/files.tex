\textbf{Changing file attributes}:
\begin{compactenum}
	\item [\symbolcoreutils] \commandcoreutils{chgrp} changes group ownership:
	\item [\texttt{h}] affects symlinks instead of referenced files,

	\item [\symbolcoreutils] \commandcoreutils{chmod} changes permissions of a file:
	\item [\texttt{ugoa}] of the owner, group, other or all users,
	\item [\texttt{+-=}] adds, removes or sets selected file mode bits,
	\item [\texttt{rwx}] selects file mode bits: read 4/write 2/execute 1.
	
	\item [\symbolcoreutils] \commandcoreutils{chown} changes owner of a file:
	\item [\texttt{R}] recursively (applies for chgrp, chmod, chown).

	\item [\symbolcoreutils] \commandcoreutils{touch} changes file timestamps:
	\item [\texttt{a}] only the access time,
	\item [\texttt{m}] only the modification time,
	\item [\texttt{c}] creates no new files,
	\item [\texttt{r}] uses another file's times as a reference,
	\item [\texttt{t}] uses custom stamp instead of current time.
\end{compactenum}

\textbf{SELinux}:
\begin{compactenum}
	\item [\symbolcoreutils] \commandcoreutils{chcon}: \dotfill ????
	\item [\symbolcoreutils] \commandcoreutils{runcon}: \dotfill ????
\end{compactenum}

\textbf{Summarizing files}:
\begin{compactenum}
	\item [\symbolcoreutils] \commandcoreutils{wc}: counts...
	\item [\texttt{l}] lines,
	\item [\texttt{w}] words,
	\item [\texttt{c}] bytes,
	\item [\texttt{m}] characters,
	\item [\texttt{L}] characters in longest line.
	\item [\symbolcoreutils] \commandcoreutils{b2sum}: \dotfill ????
	\item [\symbolcoreutils] \commandcoreutils{cksum}: CRC checksums \dotfill ????
	\item [\symbolcoreutils] \commandcoreutils{md5sum}: \dotfill ????
	\item [\symbolcoreutils] \commandcoreutils{sha1sum}: \dotfill ????
	\item [\symbolcoreutils] \commandcoreutils{sha224sum}: \dotfill ????
	\item [\symbolcoreutils] \commandcoreutils{sha256sum}: \dotfill ????
	\item [\symbolcoreutils] \commandcoreutils{sha384sum}: \dotfill ????
	\item [\symbolcoreutils] \commandcoreutils{sha512sum}: \dotfill ????
	\item [???] \textbf{shasum} prints or checks SHA message digests:
	\item [\texttt{a}] algorithm: 1/224/256/384/512/512224/512256,
	\item [\texttt{b}] reads in binary mode,
	\item [\texttt{c}] verifies SHA sums read from a file.
	\item [\symbolcoreutils] \commandcoreutils{sum}: \dotfill ????
\end{compactenum}

\textbf{Basic operations}:
\begin{compactenum}
	\item [\symbolcoreutils] \commandcoreutils{cp} copies files and directories:
	\item [\texttt{b}] makes a backup of existing destination files,
	\item [\texttt{f}] removes existing destination files if needed,
	\item [\texttt{i}] prompts before overwriting,
	\item [\texttt{n}] does not overwrite existing files,
%	\item [\texttt{a}] never follows symlinks, preserves all attributes,
	\item [\texttt{L}] always follows symlinks in ``source'',
	\item [\texttt{P}] never follows symlinks in ``source'',
	\item [\texttt{p}] preserves timestamps, mode, ownership,
	\item [\texttt{r}] copies directories recursively,
	\item [\texttt{l}] hard links files instead,
	\item [\texttt{s}] makes symbolic links instead,
	\item [\texttt{t}] copies all ``source'' arguments into ``directory'',
	\item [\texttt{T}] treats ``destination'' as a normal file,
	\item [\texttt{u}] copies only newer source files,
	\item [\texttt{v}] explains what is being done.

	\item [\symbolcoreutils] \commandcoreutils{dd} converts and copies files (non-std syntax!):
	\item [] \texttt{if=} reads from a file,
	\item [] \texttt{of=} writes to a file,
	\item [] \texttt{bs=} up to ... bytes at a time,
	\item [] \texttt{count=} copies only ... input blocks.

	\item [\symbolcoreutils] \commandcoreutils{install}: \dotfill ????

	\item [\symbolcoreutils] \commandcoreutils{tee} duplicates pipe content: % (named after the T-splitter used in plumbing) 
	\item [\texttt{a}] appends to the given files, does not overwrite,
	\item [\texttt{i}] ignores interrupts.
	
	\item [\symbolcoreutils] \commandcoreutils{mv} moves (renames) files:
	\item [\texttt{f}] does not prompt before overwriting,
	\item [] Some cp options are allowed: \texttt{b}, \texttt{i}, \texttt{n}, \texttt{t}, \texttt{T}, \texttt{u}, \texttt{v}.

	\item [\symbolcoreutils] \commandcoreutils{rm} removes files or directories:
	\item [\texttt{f}] never prompts,
	\item [\texttt{i}] always prompts,
	\item [\texttt{r}] removes directories and their contents.

	\item [\symbolcoreutils] \commandcoreutils{shred}: overwrites repeatedly.
\end{compactenum}

\textbf{Working with special file types}:
\begin{compactenum}
	\item [\symbolcoreutils] \commandcoreutils{mkdir} makes directories:
	\item [\texttt{p}] with parents as needed,
	\item [\texttt{v}] verbosely.

	\item [\symbolcoreutils] \commandcoreutils{rmdir} removes empty directories.

	\item [\symbolcoreutils] \commandcoreutils{mktemp} creates temporary files:
	\item [\texttt{d}] creates directory instead.

	\item [\symbolcoreutils] \commandcoreutils{ln} makes hard links between files:
	\item [\texttt{s}] makes soft (symbolic) links instead.

	\item [\symbolcoreutils] \commandcoreutils{unlink} \dotfill ????
	% \item [???] \textbf{unlink} calls the unlink function. USE RM INSTEAD!

	\item [\symbolcoreutils] \commandcoreutils{readlink} resolves canonical file names
	\item [\texttt{m}] by following every symlink recursively,
	\item [\texttt{f}] same and all but last part of path must exist,
	\item [\texttt{e}] same and all parts of path must exist.
	
	\item [\symbolcoreutils] \commandcoreutils{link} \dotfill ????

	\item [\symbolcoreutils] \commandcoreutils{mkfifo} \dotfill ????

	\item [\symbolcoreutils] \commandcoreutils{mknod} \dotfill ????
\end{compactenum}

\textbf{Disk usage}:
\begin{compactenum}
	\item [\symbolcoreutils] \commandcoreutils{df} reports file system disk space usage:
	\item [\texttt{h}] prints size in powers of 1024,
	\item [\texttt{i}] list inode information instead of block usage,
	\item [\texttt{t}] limits listing to file systems of given type,
	\item [\texttt{x}] limits listing to file systems not of given type,
	\item [\texttt{T}] prints file systems types.

	\item [\symbolcoreutils] \commandcoreutils{du} estimates file space usage:
	\item [\texttt{a}] writes counts for all files, not just directories,
	\item [\texttt{c}] produces a grand total,
	\item [\texttt{d}] the depth at which summing should occur,
	\item [\texttt{h}] prints sizes in human readable format,
	\item [\texttt{s}] diplays only a total,
	\item [\texttt{X}] excludes files that match pattern from pattern.

	\item [\symbolcoreutils] \commandcoreutils{stat} displays file status:
	\item [\texttt{L}] following links,
	\item [\texttt{t}] in terse form.

	\item [\symbolcoreutils] \commandcoreutils{sync} \dotfill ????

	\item [\symbolcoreutils] \commandcoreutils{truncate} \dotfill ????
\end{compactenum}

\begin{compactenum}
	\item [???] \textbf{file} determines file type.
	\item [\texttt{b}] doesn't prepend filenames to output (brief),
	\item [\texttt{i}] prints MIME, not human-readable types,
	\item [\texttt{k}] doesn't stop at the first match (keep going).
\end{compactenum}

\begin{compactenum}
	\item [???] \textbf{find} searches for files in a directory hierarchy.
	% TODO!
	[???] See also: \textbf{whereis}.
	[???] See also: \textbf{locate}, whose database can be updated with \textbf{updatedb}.
\end{compactenum}

\begin{compactenum}
	\item [???] \textbf{fsck} checks and repairs a Linux filesystem:
	\item [\texttt{a}] automatically repairs (without any question!),
	\item [\texttt{A}] tries to check all filesystems in one run,
	\item [\texttt{M}] skips mounted filesystems,
	\item [\texttt{R}] skips the root filesystem.
	\item [\texttt{t}] specifies the type(s) of filesystem to be checked,
\end{compactenum}

\textbf{Directory listing}:
\begin{compactenum}
	\item [\symbolcoreutils] \commandcoreutils{ls} lists directory contents:
	\item [\texttt{a}] does not ignore entries starting with dot, 
	\item [\texttt{F}] appends indicator to entries, 
	\item [\texttt{h}] prints human readable sizes, 
	\item [\texttt{i}] prints the index number of each file, 
	\item [\texttt{l}] prints permissions, number of hard links, owner, group, size, last-modified date as well, 
	\item [\texttt{r}] reverses order while sorting,
	\item [\texttt{R}] lists subdirectories recursively, 
	\item [\texttt{S}] sorts by file size (largest first), 
	\item [\texttt{t}] sorts by modification time (newest first), 

	\item [\symbolcoreutils] \commandcoreutils{pwd} is also a Bash builtin.

	\item [\symbolcoreutils] \commandcoreutils{dir}: \dotfill ???

	\item [\symbolcoreutils] \commandcoreutils{vdir}: \dotfill ???

	\item [\symbolcoreutils] \commandcoreutils{dircolors}: \dotfill ???

	\item [???] \textbf{tree} lists tree-like contents of directories.
	\item [\texttt{d}] skips files
	\item [\texttt{L}] limits search depth
	\item [\texttt{s}] prints sizes as well
	% stty tty???
\end{compactenum}

\textbf{Filename manipulation}:
\begin{compactenum}
	\item [\symbolcoreutils] \commandcoreutils{basename} prints path without parent directory

	\item [\symbolcoreutils] \commandcoreutils{dirname} prints only parent directory

	\item [\symbolcoreutils] \commandcoreutils{pathchk}: \dotfill ???

	\item [\symbolcoreutils] \commandcoreutils{realpath}: \dotfill ???
\end{compactenum}

\begin{compactenum}
	\item [???] \textbf{mount} mounts a filesystem.
	\item [\texttt{t}] indicates filesystem type (ext4, xfs, btrfs, ...)
\end{compactenum}

\begin{compactenum}
	\item [???] \textbf{tar} manipulates archives (``Tape ARchive''):
	\item [\texttt{f}] archive to manipulate,
	\item [\texttt{v}] verbosely lists files processed,
	\item [\texttt{c}] creates a new archive,
	\item [\texttt{r}] appends files to the end of an archive,
	\item [\texttt{t}] lists the contents of an archive,
	\item [\texttt{x}] extracts files,
	\item [\texttt{k}] doesn't replace existing files when extracting,
	\item [\texttt{a}] guesses compression program from filename,
	\item [\texttt{j}] uses compression program bzip2,
	\item [\texttt{J}] uses compression program xz,
	\item [\texttt{z}] uses compression program gzip,
	\item [\texttt{C}] changes directory before operations.
\end{compactenum}
