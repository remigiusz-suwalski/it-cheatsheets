\section{Programming in Bash}
\subsection{Shebang}
The shebang (\mintinline{bash}{#!}) at the head of a script indicates an interpreter for execution, as in \mintinline{bash}{#!/bin/bash}.
Lines starting with a \mintinline{bash}{#} (with the exception of shebang) are comments and thus won't be executed.
Even better is \mintinline{bash}{#!/usr/bin/env bash} (overflow-16365130).

\subsection{Quoting and literals}
Inside \textbf{single quotes} \mintinline{bash}{' '} nothing is interpreted: they preserve literal values of characters enclosed within them.
A single (strong) quote may not appear between single quotes, even when escaped, but it may appear between \textbf{double (weak) quotes} \mintinline{bash}{""}.

They work quite similarly, with an exception that the shell expands any variables that appear within them (pathname expansion, process substitution and word splitting are disabled, everything else works!).

It is a very important concept: without them Bash splits lines into words at whitespace characters -- tabs and spaces.
See also: \mintinline{bash}{IFS} variable and \mintinline{bash}{$'...'} and \mintinline{bash}{$"..."} (both are Bash extensions, to be done).

Avoid backticks \mintinline{bash}{`code`} at all cost!

\subsection{Variables}
\textbf{Variable} names are case sensitive.
They can contain digits and underscores as well,
but a name starting with a digit is not allowed.
Example:
\begin{minted}{bash}
var="kind"
echo ${var}ness # kindness
\end{minted}

Special variables:
\begin{compactenum}
    \item \mintinline{bash}{$0}: name of the script itself.
    \item \mintinline{bash}{$1}, \mintinline{bash}{$2}, \ldots: the first, second, etc. argument.
    \mintinline{bash}{shift} removes first argument and advances rest of them forward.
    \item \mintinline{bash}{$*} and \mintinline{bash}{$@} denote all positional parameters (\mintinline{bash}{$*}: as a single word, \mintinline{bash}{$@}: as separate strings).
    \item \mintinline{bash}{$#}: the number of positional parameters.
    \item \mintinline{bash}{$?}: exit status of last executed command.
    \item \mintinline{bash}{$$}: the process ID of the shell.
    \item \mintinline{bash}{$!}: the process ID of last executed command.
\end{compactenum}

\subsection{Expansions}
Eight expansions happen after splitting command into tokens, in the order that they are expanded.

\subsubsection{Brace expansion}
Brace expansion is used when we need to generate all possible string combinations.
Both of the commands produce the same output:
\begin{minted}{bash}
echo {I,really,love,dots}.
echo I. really. love. dots.
\end{minted}

\textbf{Warning}: it does not expand the variables (\texttt{\$var}), which is done later, but supports ranges (sequences) of characters:
\begin{minted}{bash}
echo {a..t}
a b c d e f g h i j k l m n o p q r s t
\end{minted}

and (maybe zero paded or with an increment rate) integers, assuming the Bash version is 4 or newer:

\begin{minted}{bash}
echo {01..10..1}.
01. 02. 03. 04. 05. 06. 07. 08. 09. 10.
\end{minted}

\subsubsection{Tilde expansion}
There is a \textbf{tilde expansion} as well.
The expression \mintinline{bash}{~[user]} expands to the home directory of the current (or given) user.

\subsubsection{Parameter expansion}
\begin{enumerate}
\item \mintinline{bash}{${var^}}, \mintinline{bash}{${var,}}, \mintinline{bash}{${var~}} convert first character to upper, lower or reverse case (every, when caret/comma/tilde are doubled).
This has an impact on every array element.

\item \mintinline{bash}{${var#pattern}}, \mintinline{bash}{${var%pattern}} remove the pattern from the beginning, end of variable (greedy when hash, percent sign are doubled).
Application: extracting parts of a filename.

\item \mintinline{bash}{${var/pattern/string}} performs a single (when first slash is doubled: full) search and replace operation.

\item \mintinline{bash}{${#var}} returns length of the string.

\item \mintinline{bash}{${var:offset:length}} skips first \texttt{offset} characters from \texttt{var} and truncates the output to given length.
\texttt{:length} may be skipped.

Negative values separated with extra space are accepted.

\item \mintinline{bash}{${var:-value}} uses a default value, if \texttt{var} is empty or unset.

\mintinline{bash}{${var:=value}} does the same, but performs an assignment as well.

\mintinline{bash}{${var:+value}} uses an alternative value if \texttt{var} isn't empty or unset!
\end{enumerate}

\subsubsection{Command substitution}
To execute commands in a subshell and then pass their stdout (but not stderr!), use \mintinline{bash}{$( commands )}.

(To group them without new shell, use \mintinline{bash}{{ cmds }}.)

\subsubsection{Arithmetic expansion}
The arithmetic expression \mintinline{bash}{$(( ... ))} is evaluated and expands to the result.
Bash guarantees that the output will be a one-word integer.
Exit code is zero $\iff$ output is nonzero.

\subsubsection{Process substitution}
This kind of substitution: \mintinline{bash}{<( ... )} and \mintinline{bash}{>( ... )} (not specified by POSIX!), where input or output of a command appears as a temporary file, is performed simultaneously with the following: arithmetic and parameter expansions, command substitution.

\subsubsection{Word splitting}

\subsubsection{File name expansion}
Glob patterns (composed of normal characters and \texttt{*}, \texttt{?}, \texttt{[...]}) are not regular expressions!
Bash supports extended globs too (\texttt{X(list)} where X is one from \texttt{?*+@!}).

\subsection{Streams}
There are always three default files open:
\begin{compactenum}
\item \emph{stdin} (the keyboard, file descriptor 0),
\item \emph{stdout} (the screen, file descriptor 1) and
\item \emph{stderr} (error messages, file descriptor 2).
\end{compactenum}

These \textbf{streams} can be \textbf{redirected}:
\begin{compactenum}
\item \mintinline{bash}{cmd > file} redirects to a file (overwrites),
\item \mintinline{bash}{cmd >> file} appends instead,
\item \mintinline{bash}{m>n} (\mintinline{bash}{m>&n}) redirects a file descriptor to a file
(or another file descriptor),
\item \mintinline{bash}{&>file} redirects stdout and stderr to a file,
% \item \texttt{:> file} truncates file to zero length,
\item \mintinline{bash}{|} (pipe) serves as a command chaining tool.
\end{compactenum}

\textbf{Here document} is a section of a source code file that is treated as if it were a separate file:

\begin{minted}{bash}
cat <<EOF > /path/to/your/file
   Line to be written in your file.
EOF
\end{minted}

Using \mintinline{bash}{'EOF'} instead prevents variable expansion.

\subsection{Control flow statements}
The one-line constructs \texttt{\&\&} and \texttt{||} work not like and, or ($\wedge$, $\vee$), but the \emph{if -- then -- else} statement.
Nonzero exit status denotes (usually) a failure.

\subsubsection{Loops}
\begin{minted}{bash}
while read myline; do
  echo "It says ${myline}"
done < some_file

for var in "the first" "the second"; do
  echo "${var}"
done
for (( i = 1; i <= 10; i++ )); do
  echo "i = ${i}."
done # C-style
\end{minted}
\subsubsection{Conditionals}
Here at least one statement must be specified inside every block,
but one can use a single colon (:) as a null statement to avoid
rewriting the code.

\begin{minted}{bash}
if condition; then
  commands
elif second_condition; then
  some_commands
else
  other_commands
fi

select word in "Bash" "Haskell" "Python"
do
  echo "Your language is $word".
done
\end{minted}

There is also a case instruction:
\begin{minted}{bash}
case $language in
  bash)
    echo "Bourne Again Shell!"
  ;;
  python|haskell)
    echo "Python or Haskell!"
  ;;
  *)
    echo "Unknown language!"
  ;; # optional
esac
\end{minted}

\subsubsection{Single versus double square brackets}
The \texttt{[} command is defined by POSIX, \texttt{]} prevents its further arguments from being used.
Double brackets, the \texttt{[[} command, are a Bash extension that changes the way text ist being parsed:
\begin{compactenum}
\item \texttt{<} and \texttt{>} compare lexicographically instead of redirecting streams (inside \texttt{[ ]} they have to be escaped: \texttt{\textbackslash{}<}, \texttt{\textbackslash{}>})
\item \texttt{||} and \texttt{\&\&} are true logical operators.
\item \texttt{( )} groups commands (not a subshell!)
\item \texttt{==} without quotes does pattern matching
\item \texttt{=$\sim$} matches extended regex (no \texttt{[]}-alternative)
\end{compactenum}

\subsubsection{Testing conditions}
All of these tests but \texttt{-h} follow symlinks (!), beware.

\begin{compactenum}
\item \textbf{File tests}:
\begin{compactenum}
    \item \texttt{-e} file exists,
    \texttt{-s} file is nonempty,
    \item \texttt{-d} directory,
    \texttt{-f} regular file,
    \texttt{-h} symlink,
    \item \texttt{-b} block device,
    \texttt{-c} character device,
    \item \texttt{-p} named pipe,
    \texttt{-S} socket.
\end{compactenum}
\item \textbf{File permissions}:
\begin{compactenum}
    \item \texttt{-r} readable,
    \texttt{-w} writable,
    \texttt{-x} executable,
    \item \texttt{-u} setuid,
    \texttt{-g} setgid,
    \texttt{-k} sticky bit.
\end{compactenum}
\item \textbf{String tests}: \texttt{-z} empty, \texttt{-n} nonempty.
\item \textbf{Arithmetic tests}:
  \texttt{-eq} $=$,
  \texttt{-lt} $<$,
  \texttt{-gt} $>$,
  \texttt{-ne} $\neq$,
  \texttt{-le} $\le$,
  \texttt{-ge} $\ge$.
\end{compactenum}

Both \texttt{break} and \texttt{continue} are accepted.

\subsection{builtins -- Built-in objects}
To be done ....

%


%